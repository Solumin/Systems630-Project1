%%%%%%%%%%%%%%%%%%%%%%%%%%%%%%%%%%%%%%%%%
% University/School Laboratory Report
% LaTeX Template
% Version 3.0 (4/2/13)
%
% This template has been downloaded from:
% http://www.LaTeXTemplates.com
%
% Original author:
% Linux and Unix Users Group at Virginia Tech Wiki 
% (https://vtluug.org/wiki/Example_LaTeX_chem_lab_report)
%
% License:
% CC BY-NC-SA 3.0 (http://creativecommons.org/licenses/by-nc-sa/3.0/)
%
%%%%%%%%%%%%%%%%%%%%%%%%%%%%%%%%%%%%%%%%%

%----------------------------------------------------------------------------------------
%	PACKAGES AND DOCUMENT CONFIGURATIONS
%----------------------------------------------------------------------------------------

\documentclass{article}

\usepackage{mhchem} % Package for chemical equation typesetting
\usepackage{siunitx} % Provides the \SI{}{} command for typesetting SI units
\usepackage{amsmath}
\usepackage{amssymb}
\usepackage{listings}
%\usepackage{lstcoq}
\usepackage{fullpage}
\usepackage{hyperref}
\usepackage{graphicx} % Required for the inclusion of images

\setlength\parindent{0pt} % Removes all indentation from paragraphs

\renewcommand{\labelenumi}{\alph{enumi}.} % Make numbering in the enumerate environment by letter rather than number (e.g. section 6)

%\usepackage{times} % Uncomment to use the Times New Roman font

%----------------------------------------------------------------------------------------
%	DOCUMENT INFORMATION
%----------------------------------------------------------------------------------------

\title{Project 1 - Report} % Title

\author{Cibele \textsc{Freire} and Theodore \textsc{Sudol}} % Author name

\date{\today} % Date for the report
\newcommand{\link}[3]{%
    \underline{\texttt{\href{#1:#2}{#3}}}%
}

\begin{document}

%\lstset{language=Coq} probably should set this to typescript

\maketitle % Insert the title, author and date

\begin{center}
\begin{tabular}{l r}
%Date Performed: & January 1, 2012 \\ % Date the experiment was performed
%Partners: & James Smith \\ % Partner names
%& Mary Smith \\
Instructor: & Professor Emery Berger % Instructor/supervisor
\end{tabular}
\end{center}



% If you wish to include an abstract, uncomment the lines below
%\begin{abstract}
%This document is In this document we describe how the development of this system was conducted. The choices that were made 
%\end{abstract}


\section{Introduction}\label{intro}

\paragraph{}
The project consists of designing and implementing an interpreter for Python bytecode in TypeScript, a typed superset of JavaScript. The interpreter is designed to work in the browser, loading and executing programs from Python bytecode files, i.e. .pyc files. It can also be run with the Node.JS framework.

\paragraph{}
In Section~\ref{parser} we explain how we parsed the bytecode, what is the Marshall format and what data structures we used. In Section~\ref{interpreter} we explain the design of the interpreter and what we were able to implement.

\section{Parser}\label{parser}

\paragraph{}
The bytecode generated by a Python compiler is stored in Marshall format. Each .pyc file is composed of:

\begin{itemize}
	\item Magic number: consist of the first 8 bytes of the .pyc file and it indicates from which Python version the bytecode was generated;
	\item Date: the next 8 bytes and it indicates the date of compilation;
	\item Code object: the remaining of the file. More details on Section~\ref{code object}
\end{itemize}

We define the class Unmarshaller to handle the parsing phase. The class has the attributes index, input, magicNumber, date, internedStrs, and output.

\begin{itemize}
	\item \texttt{index}: it keeps track of which byte of the file is being currently read 
	\item \texttt{input}: the path for the .pyc file that is going to be processed
	\item \texttt{magicNumber}: stores the magic number
	\item \texttt{date}: stores the date of compilation
	\item \texttt{internedStrs}:
	\item \texttt{output}: the code object generated after parsing the .pyc file
\end{itemize}

The following are the central methods implemented by this class.

\begin{itemize}
	\item \texttt{value()}:
	\item \texttt{unmarshallCodeString()}:
	\item \texttt{unmarshal()}:
\end{itemize}

\subsection{Code objects}\label{code object}

\paragraph{}
A code object is a representation of a block of executable Python code. These can include functions, modules and classes. Of course, a block of code is more than just a sequence of opcodes. Each code object includes pieces of meta data, such as the name of the function and the number of arguments. There's also information that's useful for debugging, like the first line number of the code relative to the original source file. All of this data is included in the .pyc file.

\paragraph{}
We designed our code object class (Py\_CodeObject) around the official Python documentation, with particular inspiration from the \texttt{inspect} library. Due the limitations of our interpreter, not all of the fields were needed. For example, the \texttt{co\_cellvars} field is used for closures, which we did not implement. However, we included these fields for the sake of completeness and compatibility with the Python design.

\paragraph{}
Code objects are primary used in the interpreter as components of frame objects, which are discussed in section~\ref{py-frameobject}.

\section{Interpreter}\label{interpreter}

\paragraph{}
Our Python interpreter is based around a simple Fetch-Decode-Execute loop. In each cycle, the interpreter reads an opcode, finds the appropriate function, and executes that function. Python bytecode uses a stack-based design, but the interpreter does not maintain its own stack. Instead, frame objects are used to maintain interpreter state.

\subsection{Frame Objects}\label{py-frameobject}


\texttt{Interpreter} 

\subsection{Numeric methods}

Issues with numeric methods and how they were fixed.

\subsection{Not implemented}

\subsection{Using the browser}



%\bibliographystyle{unsrt}
%\bibliography{sample}

%----------------------------------------------------------------------------------------


\end{document}
